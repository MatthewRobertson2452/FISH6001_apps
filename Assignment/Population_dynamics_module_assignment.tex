% Options for packages loaded elsewhere
\PassOptionsToPackage{unicode}{hyperref}
\PassOptionsToPackage{hyphens}{url}
%
\documentclass[
  12pt,
]{article}
\usepackage{amsmath,amssymb}
\usepackage{iftex}
\ifPDFTeX
  \usepackage[T1]{fontenc}
  \usepackage[utf8]{inputenc}
  \usepackage{textcomp} % provide euro and other symbols
\else % if luatex or xetex
  \usepackage{unicode-math} % this also loads fontspec
  \defaultfontfeatures{Scale=MatchLowercase}
  \defaultfontfeatures[\rmfamily]{Ligatures=TeX,Scale=1}
\fi
\usepackage{lmodern}
\ifPDFTeX\else
  % xetex/luatex font selection
\fi
% Use upquote if available, for straight quotes in verbatim environments
\IfFileExists{upquote.sty}{\usepackage{upquote}}{}
\IfFileExists{microtype.sty}{% use microtype if available
  \usepackage[]{microtype}
  \UseMicrotypeSet[protrusion]{basicmath} % disable protrusion for tt fonts
}{}
\makeatletter
\@ifundefined{KOMAClassName}{% if non-KOMA class
  \IfFileExists{parskip.sty}{%
    \usepackage{parskip}
  }{% else
    \setlength{\parindent}{0pt}
    \setlength{\parskip}{6pt plus 2pt minus 1pt}}
}{% if KOMA class
  \KOMAoptions{parskip=half}}
\makeatother
\usepackage{xcolor}
\usepackage[margin=1in]{geometry}
\usepackage{color}
\usepackage{fancyvrb}
\newcommand{\VerbBar}{|}
\newcommand{\VERB}{\Verb[commandchars=\\\{\}]}
\DefineVerbatimEnvironment{Highlighting}{Verbatim}{commandchars=\\\{\}}
% Add ',fontsize=\small' for more characters per line
\usepackage{framed}
\definecolor{shadecolor}{RGB}{248,248,248}
\newenvironment{Shaded}{\begin{snugshade}}{\end{snugshade}}
\newcommand{\AlertTok}[1]{\textcolor[rgb]{0.94,0.16,0.16}{#1}}
\newcommand{\AnnotationTok}[1]{\textcolor[rgb]{0.56,0.35,0.01}{\textbf{\textit{#1}}}}
\newcommand{\AttributeTok}[1]{\textcolor[rgb]{0.13,0.29,0.53}{#1}}
\newcommand{\BaseNTok}[1]{\textcolor[rgb]{0.00,0.00,0.81}{#1}}
\newcommand{\BuiltInTok}[1]{#1}
\newcommand{\CharTok}[1]{\textcolor[rgb]{0.31,0.60,0.02}{#1}}
\newcommand{\CommentTok}[1]{\textcolor[rgb]{0.56,0.35,0.01}{\textit{#1}}}
\newcommand{\CommentVarTok}[1]{\textcolor[rgb]{0.56,0.35,0.01}{\textbf{\textit{#1}}}}
\newcommand{\ConstantTok}[1]{\textcolor[rgb]{0.56,0.35,0.01}{#1}}
\newcommand{\ControlFlowTok}[1]{\textcolor[rgb]{0.13,0.29,0.53}{\textbf{#1}}}
\newcommand{\DataTypeTok}[1]{\textcolor[rgb]{0.13,0.29,0.53}{#1}}
\newcommand{\DecValTok}[1]{\textcolor[rgb]{0.00,0.00,0.81}{#1}}
\newcommand{\DocumentationTok}[1]{\textcolor[rgb]{0.56,0.35,0.01}{\textbf{\textit{#1}}}}
\newcommand{\ErrorTok}[1]{\textcolor[rgb]{0.64,0.00,0.00}{\textbf{#1}}}
\newcommand{\ExtensionTok}[1]{#1}
\newcommand{\FloatTok}[1]{\textcolor[rgb]{0.00,0.00,0.81}{#1}}
\newcommand{\FunctionTok}[1]{\textcolor[rgb]{0.13,0.29,0.53}{\textbf{#1}}}
\newcommand{\ImportTok}[1]{#1}
\newcommand{\InformationTok}[1]{\textcolor[rgb]{0.56,0.35,0.01}{\textbf{\textit{#1}}}}
\newcommand{\KeywordTok}[1]{\textcolor[rgb]{0.13,0.29,0.53}{\textbf{#1}}}
\newcommand{\NormalTok}[1]{#1}
\newcommand{\OperatorTok}[1]{\textcolor[rgb]{0.81,0.36,0.00}{\textbf{#1}}}
\newcommand{\OtherTok}[1]{\textcolor[rgb]{0.56,0.35,0.01}{#1}}
\newcommand{\PreprocessorTok}[1]{\textcolor[rgb]{0.56,0.35,0.01}{\textit{#1}}}
\newcommand{\RegionMarkerTok}[1]{#1}
\newcommand{\SpecialCharTok}[1]{\textcolor[rgb]{0.81,0.36,0.00}{\textbf{#1}}}
\newcommand{\SpecialStringTok}[1]{\textcolor[rgb]{0.31,0.60,0.02}{#1}}
\newcommand{\StringTok}[1]{\textcolor[rgb]{0.31,0.60,0.02}{#1}}
\newcommand{\VariableTok}[1]{\textcolor[rgb]{0.00,0.00,0.00}{#1}}
\newcommand{\VerbatimStringTok}[1]{\textcolor[rgb]{0.31,0.60,0.02}{#1}}
\newcommand{\WarningTok}[1]{\textcolor[rgb]{0.56,0.35,0.01}{\textbf{\textit{#1}}}}
\usepackage{graphicx}
\makeatletter
\newsavebox\pandoc@box
\newcommand*\pandocbounded[1]{% scales image to fit in text height/width
  \sbox\pandoc@box{#1}%
  \Gscale@div\@tempa{\textheight}{\dimexpr\ht\pandoc@box+\dp\pandoc@box\relax}%
  \Gscale@div\@tempb{\linewidth}{\wd\pandoc@box}%
  \ifdim\@tempb\p@<\@tempa\p@\let\@tempa\@tempb\fi% select the smaller of both
  \ifdim\@tempa\p@<\p@\scalebox{\@tempa}{\usebox\pandoc@box}%
  \else\usebox{\pandoc@box}%
  \fi%
}
% Set default figure placement to htbp
\def\fps@figure{htbp}
\makeatother
\setlength{\emergencystretch}{3em} % prevent overfull lines
\providecommand{\tightlist}{%
  \setlength{\itemsep}{0pt}\setlength{\parskip}{0pt}}
\setcounter{secnumdepth}{-\maxdimen} % remove section numbering
\usepackage{bookmark}
\IfFileExists{xurl.sty}{\usepackage{xurl}}{} % add URL line breaks if available
\urlstyle{same}
\hypersetup{
  pdftitle={Population dynamics module assignment},
  pdfauthor={YOUR NAME HERE},
  hidelinks,
  pdfcreator={LaTeX via pandoc}}

\title{Population dynamics module assignment}
\author{YOUR NAME HERE}
\date{SUBMISSION DATE HERE}

\begin{document}
\maketitle

\section{Data Setup}\label{data-setup}

In this assignment you are going to modify the age-structured simulation
that we used earlier in the course to match the basic biology of your
stock. There should be a CSAS report or other stock assessment document
that contains much of this information, if it exists for your stock. If
it doesn't exist in a government report, I would like you to find a
different citation for the biological information for your stock. If
there is no research on your particular stock for the specified life
history information, then you can use Fishbase or some other source that
describes that information for your species. Finally, if there is no
research on your species for a life history trait, you can give it your
best guess, but you will need to indicate this and I will search to see
if there is truly no research on your species for that trait.

Similar to the example in class, the simulation will run for 50 years,
lets say from 2000 to 2050.

\begin{Shaded}
\begin{Highlighting}[]
\NormalTok{yrs}\OtherTok{\textless{}{-}}\FunctionTok{seq}\NormalTok{(}\AttributeTok{from=}\DecValTok{2000}\NormalTok{, }\AttributeTok{to=}\DecValTok{2050}\NormalTok{)}
\NormalTok{nyears}\OtherTok{\textless{}{-}}\FunctionTok{length}\NormalTok{(yrs)}
\end{Highlighting}
\end{Shaded}

\subsection{Number of ages (1 point)}\label{number-of-ages-1-point}

What is the maximum age of your stock? Provide a citation.

\begin{Shaded}
\begin{Highlighting}[]
\CommentTok{\#Change the value to match your stock\textquotesingle{}s value}
\NormalTok{nages}\OtherTok{\textless{}{-}}\DecValTok{15}
\end{Highlighting}
\end{Shaded}

\subsection{Maximum length (1 points)}\label{maximum-length-1-points}

What is \(L_{inf}\) for your stock? Provide a citation.

\begin{Shaded}
\begin{Highlighting}[]
\CommentTok{\#Change the value to match your stock\textquotesingle{}s value}
\NormalTok{linf}\OtherTok{\textless{}{-}}\DecValTok{100}
\end{Highlighting}
\end{Shaded}

\begin{Shaded}
\begin{Highlighting}[]
\NormalTok{laa}\OtherTok{\textless{}{-}}\FunctionTok{rep}\NormalTok{(}\ConstantTok{NA}\NormalTok{, nages)}
\ControlFlowTok{for}\NormalTok{(i }\ControlFlowTok{in} \DecValTok{1}\SpecialCharTok{:}\NormalTok{nages)\{}
\NormalTok{laa[i]}\OtherTok{\textless{}{-}}\NormalTok{linf}\SpecialCharTok{*}\NormalTok{(}\DecValTok{1}\SpecialCharTok{{-}}\FunctionTok{exp}\NormalTok{(}\SpecialCharTok{{-}}\NormalTok{(}\DecValTok{5}\SpecialCharTok{/}\NormalTok{nages)}\SpecialCharTok{*}\NormalTok{(i}\SpecialCharTok{{-}}\NormalTok{(linf}\SpecialCharTok{/}\DecValTok{1000}\NormalTok{))))}
\NormalTok{\}}

\FunctionTok{plot}\NormalTok{(laa}\SpecialCharTok{\textasciitilde{}}\FunctionTok{seq}\NormalTok{(}\AttributeTok{from=}\DecValTok{1}\NormalTok{, }\AttributeTok{to=}\NormalTok{nages, }\AttributeTok{by=}\DecValTok{1}\NormalTok{), }\AttributeTok{pch=}\DecValTok{19}\NormalTok{, }\AttributeTok{xlab=}\StringTok{"Age"}\NormalTok{,}
     \AttributeTok{ylab=}\StringTok{"Mean Length (cm)"}\NormalTok{, }\AttributeTok{las=}\DecValTok{1}\NormalTok{)}
\FunctionTok{lines}\NormalTok{(laa}\SpecialCharTok{\textasciitilde{}}\FunctionTok{seq}\NormalTok{(}\AttributeTok{from=}\DecValTok{1}\NormalTok{, }\AttributeTok{to=}\NormalTok{nages, }\AttributeTok{by=}\DecValTok{1}\NormalTok{))}
\end{Highlighting}
\end{Shaded}

\pandocbounded{\includegraphics[keepaspectratio]{Population_dynamics_module_assignment_files/figure-latex/lplot-1.pdf}}

\subsection{Length-weight relationship (1
point)}\label{length-weight-relationship-1-point}

What is the allometric growth coefficient (\(b\)) for your stock?
Provide a citation.

\begin{Shaded}
\begin{Highlighting}[]
\CommentTok{\#Change the value to match your stock\textquotesingle{}s value}
\NormalTok{b}\OtherTok{\textless{}{-}}\DecValTok{3}
\end{Highlighting}
\end{Shaded}

\begin{Shaded}
\begin{Highlighting}[]
\NormalTok{waa}\OtherTok{\textless{}{-}}\FunctionTok{rep}\NormalTok{(}\ConstantTok{NA}\NormalTok{, nages)}
\ControlFlowTok{for}\NormalTok{(i }\ControlFlowTok{in} \DecValTok{1}\SpecialCharTok{:}\NormalTok{nages)\{}
\NormalTok{  waa[i]}\OtherTok{\textless{}{-}}\NormalTok{(}\FloatTok{0.001}\SpecialCharTok{*}\NormalTok{laa[i]}\SpecialCharTok{\^{}}\NormalTok{b)}\SpecialCharTok{/}\DecValTok{100}
\NormalTok{  \}}

\FunctionTok{par}\NormalTok{(}\AttributeTok{mfrow=}\FunctionTok{c}\NormalTok{(}\DecValTok{2}\NormalTok{,}\DecValTok{1}\NormalTok{))}
\FunctionTok{plot}\NormalTok{(waa}\SpecialCharTok{\textasciitilde{}}\NormalTok{laa, }\AttributeTok{pch=}\DecValTok{19}\NormalTok{, }\AttributeTok{xlab=}\StringTok{"Mean Length (cm)"}\NormalTok{, }\AttributeTok{ylab=}\StringTok{"Mean Weight (kg)"}\NormalTok{,}
     \AttributeTok{las=}\DecValTok{1}\NormalTok{, }\AttributeTok{xlim=}\FunctionTok{c}\NormalTok{(}\FunctionTok{min}\NormalTok{(laa)}\SpecialCharTok{{-}}\DecValTok{1}\NormalTok{,}\FunctionTok{max}\NormalTok{(laa)}\SpecialCharTok{+}\DecValTok{1}\NormalTok{), }\AttributeTok{ylim=}\FunctionTok{c}\NormalTok{(}\DecValTok{0}\NormalTok{,}\FunctionTok{max}\NormalTok{(waa)))}
\FunctionTok{lines}\NormalTok{(waa}\SpecialCharTok{\textasciitilde{}}\NormalTok{laa)}
\FunctionTok{plot}\NormalTok{(waa}\SpecialCharTok{\textasciitilde{}}\FunctionTok{seq}\NormalTok{(}\AttributeTok{from=}\DecValTok{1}\NormalTok{, }\AttributeTok{to=}\NormalTok{nages, }\AttributeTok{by=}\DecValTok{1}\NormalTok{), }\AttributeTok{pch=}\DecValTok{19}\NormalTok{, }\AttributeTok{xlab=}\StringTok{"Age"}\NormalTok{, }\AttributeTok{ylab=}\StringTok{"Mean Weight"}\NormalTok{,}
     \AttributeTok{las=}\DecValTok{1}\NormalTok{)}
\FunctionTok{lines}\NormalTok{(waa}\SpecialCharTok{\textasciitilde{}}\FunctionTok{seq}\NormalTok{(}\AttributeTok{from=}\DecValTok{1}\NormalTok{, }\AttributeTok{to=}\NormalTok{nages, }\AttributeTok{by=}\DecValTok{1}\NormalTok{))}
\end{Highlighting}
\end{Shaded}

\pandocbounded{\includegraphics[keepaspectratio]{Population_dynamics_module_assignment_files/figure-latex/weight-1.pdf}}

\subsection{Maturity (1 point)}\label{maturity-1-point}

What is the length at 50\% maturity (\(L_{50}\)) for females of your
stock? Provide a citation.

\begin{Shaded}
\begin{Highlighting}[]
\CommentTok{\#Change the value to match your stock\textquotesingle{}s value}
\NormalTok{l50}\OtherTok{\textless{}{-}}\DecValTok{70}
\end{Highlighting}
\end{Shaded}

\begin{Shaded}
\begin{Highlighting}[]
\NormalTok{maa}\OtherTok{\textless{}{-}}\FunctionTok{rep}\NormalTok{(}\ConstantTok{NA}\NormalTok{, nages)}
\ControlFlowTok{for}\NormalTok{(i }\ControlFlowTok{in} \DecValTok{1}\SpecialCharTok{:}\NormalTok{nages)\{}
\NormalTok{  maa[i]}\OtherTok{\textless{}{-}}\DecValTok{1}\SpecialCharTok{/}\NormalTok{(}\DecValTok{1}\SpecialCharTok{+}\FunctionTok{exp}\NormalTok{(}\SpecialCharTok{{-}}\FloatTok{0.5}\SpecialCharTok{*}\NormalTok{(laa[i]}\SpecialCharTok{{-}}\NormalTok{l50)))}
\NormalTok{\}}
\FunctionTok{plot}\NormalTok{(maa}\SpecialCharTok{\textasciitilde{}}\FunctionTok{seq}\NormalTok{(}\AttributeTok{from=}\DecValTok{1}\NormalTok{, }\AttributeTok{to=}\NormalTok{nages, }\AttributeTok{by=}\DecValTok{1}\NormalTok{), }\AttributeTok{pch=}\DecValTok{19}\NormalTok{, }\AttributeTok{xlab=}\StringTok{"Age"}\NormalTok{,}
     \AttributeTok{ylab=}\StringTok{"Probability of maturity"}\NormalTok{, }\AttributeTok{las=}\DecValTok{1}\NormalTok{)}
\FunctionTok{lines}\NormalTok{(maa}\SpecialCharTok{\textasciitilde{}}\FunctionTok{seq}\NormalTok{(}\AttributeTok{from=}\DecValTok{1}\NormalTok{, }\AttributeTok{to=}\NormalTok{nages, }\AttributeTok{by=}\DecValTok{1}\NormalTok{))}
\end{Highlighting}
\end{Shaded}

\pandocbounded{\includegraphics[keepaspectratio]{Population_dynamics_module_assignment_files/figure-latex/maturity-1.pdf}}

Like was discussed in lecture, we are going to assume that natural
mortality (\(M\)) is constant at 0.2. No need to change this for the
assignment.

\begin{Shaded}
\begin{Highlighting}[]
\NormalTok{M}\OtherTok{\textless{}{-}}\FunctionTok{matrix}\NormalTok{(}\FloatTok{0.2}\NormalTok{, nyears,nages)}
\end{Highlighting}
\end{Shaded}

Once all of these components for the population have been set-up, we can
start to think about the fishery.

\section{The fishery}\label{the-fishery}

For the purposes of this simulation, I am going to assume that the
fishery only targets mature fish and also has followed a specific
pattern of exploitation over time.

In this case, fishing mortality will start at 0 in 2000 and steadily
increase to a maximum value of 0.3 at which point it will level off and
the fishery will maintain at a steady amount of fishing pressure until
the end of the simulation.

None of the code in this section needs to be modified.

\begin{Shaded}
\begin{Highlighting}[]
\NormalTok{Fsel}\OtherTok{\textless{}{-}}\FunctionTok{ifelse}\NormalTok{(maa}\SpecialCharTok{\textless{}}\FloatTok{0.5}\NormalTok{, }\DecValTok{0}\NormalTok{,}\DecValTok{1}\NormalTok{)}
\FunctionTok{par}\NormalTok{(}\AttributeTok{mfrow=}\FunctionTok{c}\NormalTok{(}\DecValTok{2}\NormalTok{,}\DecValTok{1}\NormalTok{))}
\FunctionTok{plot}\NormalTok{(Fsel}\SpecialCharTok{\textasciitilde{}}\NormalTok{maa,}\AttributeTok{pch=}\DecValTok{19}\NormalTok{, }\AttributeTok{xlab=}\StringTok{"Probability of Maturity At Age"}\NormalTok{,}
     \AttributeTok{ylab=}\StringTok{"Fishery Selectivity"}\NormalTok{, }\AttributeTok{las=}\DecValTok{1}\NormalTok{)}
\FunctionTok{lines}\NormalTok{(Fsel}\SpecialCharTok{\textasciitilde{}}\NormalTok{maa)}
\FunctionTok{plot}\NormalTok{(Fsel}\SpecialCharTok{\textasciitilde{}}\NormalTok{laa,}\AttributeTok{pch=}\DecValTok{19}\NormalTok{, }\AttributeTok{xlab=}\StringTok{"Length At Age"}\NormalTok{, }\AttributeTok{ylab=}\StringTok{"Fishery Selectivity"}\NormalTok{, }
     \AttributeTok{las=}\DecValTok{1}\NormalTok{)}
\FunctionTok{lines}\NormalTok{(Fsel}\SpecialCharTok{\textasciitilde{}}\NormalTok{laa)}
\end{Highlighting}
\end{Shaded}

\pandocbounded{\includegraphics[keepaspectratio]{Population_dynamics_module_assignment_files/figure-latex/sel-1.pdf}}

\begin{Shaded}
\begin{Highlighting}[]
\NormalTok{Ftrend}\OtherTok{\textless{}{-}}\FunctionTok{rep}\NormalTok{(}\ConstantTok{NA}\NormalTok{, nyears)}
\NormalTok{incF}\OtherTok{\textless{}{-}}\FunctionTok{round}\NormalTok{(nyears}\SpecialCharTok{*}\FloatTok{0.25}\NormalTok{,}\DecValTok{0}\NormalTok{)}
\NormalTok{Ftrend[}\DecValTok{1}\SpecialCharTok{:}\NormalTok{incF]}\OtherTok{\textless{}{-}}\FunctionTok{seq}\NormalTok{(}\AttributeTok{from=}\DecValTok{0}\NormalTok{, }\AttributeTok{to=}\FloatTok{0.3}\NormalTok{, }\AttributeTok{length=}\NormalTok{incF)}
\NormalTok{Ftrend[(incF}\SpecialCharTok{+}\DecValTok{1}\NormalTok{)}\SpecialCharTok{:}\NormalTok{nyears]}\OtherTok{\textless{}{-}}\FloatTok{0.3}

\FunctionTok{plot}\NormalTok{(Ftrend}\SpecialCharTok{\textasciitilde{}}\NormalTok{yrs,}\AttributeTok{pch=}\DecValTok{19}\NormalTok{, }\AttributeTok{xlab=}\StringTok{"Year"}\NormalTok{, }\AttributeTok{ylab=}\StringTok{"Fishing Mortality"}\NormalTok{, }\AttributeTok{las=}\DecValTok{1}\NormalTok{)}
\FunctionTok{lines}\NormalTok{(Ftrend}\SpecialCharTok{\textasciitilde{}}\NormalTok{yrs)}
\end{Highlighting}
\end{Shaded}

\pandocbounded{\includegraphics[keepaspectratio]{Population_dynamics_module_assignment_files/figure-latex/fhistory-1.pdf}}

\section{Simulation}\label{simulation}

Now that the species biology and fishery have been defined, we can run a
theoretical simulation of how this fishery would impact a theoretical
population with your stock's biology.

None of the code in this section needs to be modified.

\section{Population dynamics
comparison}\label{population-dynamics-comparison}

After the simulation runs, we can examine how the population actually
changed over time. In the following sections I will ask you questions,
including to compare your stock to mine. Any comparisons should focus on
large changes rather than interannual variability. Much of the noise
will be driven by the simulation and will not be meaningful, but you
should be able to identify large changes that are driven by differences
in life-history traits.

The life-history traits for ``Matts stock'' are the same as what we
examined for the age-structured model in class. So, \(nages=15\),
\(L_{inf}=100\), \(b=3\), and \(L_{50}=70\). When describing differences
in the population dynamics, think about how our simulated fishery might
be impacting these populations differently based on the defined
life-history traits. I will mark your assignment based on thought
process rather than being correct about why a specific change occurred.

To make sure you can identify which figure is for your stock please
modify the following code to include your stock name:

\begin{Shaded}
\begin{Highlighting}[]
\CommentTok{\# modify the following code with your stocks name}
\NormalTok{stock}\OtherTok{\textless{}{-}}\StringTok{"YOURSTOCK"}
\end{Highlighting}
\end{Shaded}

\subsection{Age and growth (2 points)}\label{age-and-growth-2-points}

Examine your length-at-age and weight-at-age plots. Which age classes
contribute most to total biomass, and why (1 pt)? How does the maximum
age of your stock influence total biomass accumulation and the stock's
resilience to fishing (1 pt)?

\subsection{Maturity and Selectivity (2
points)}\label{maturity-and-selectivity-2-points}

Examine your maturity and fishery selectivity plots. Which age or size
classes are most vulnerable to the fishery (1 pt)? Right now, the
fishery only targets mature fish, how would decreasing the targeted
length or age of harvested fish modify your SSB trend (1 pt)?

\subsection{SSB and Age Structure (2
points)}\label{ssb-and-age-structure-2-points}

Compare your SSB and abundance-at-age plots to mine. How does the age
structure and growth pattern of your stock influence its sensitivity to
the fishery compared to mine (1 pt)? Which age classes act as a buffer
against declines in SSB, and why (1 pt)?''

\begin{Shaded}
\begin{Highlighting}[]
\FunctionTok{par}\NormalTok{(}\AttributeTok{mfrow=}\FunctionTok{c}\NormalTok{(}\DecValTok{2}\NormalTok{,}\DecValTok{1}\NormalTok{))}
\FunctionTok{plot}\NormalTok{(SSB}\SpecialCharTok{\textasciitilde{}}\NormalTok{yrs, }\AttributeTok{type=}\StringTok{"l"}\NormalTok{, }\AttributeTok{xlab=}\StringTok{"Year"}\NormalTok{, }\AttributeTok{ylab=}\StringTok{"Spawning Stock Biomass"}\NormalTok{, }
     \AttributeTok{las=}\DecValTok{1}\NormalTok{, }\AttributeTok{lwd=}\DecValTok{2}\NormalTok{, }\AttributeTok{main=}\NormalTok{stock)}
\FunctionTok{plot}\NormalTok{(sim\_pop\_dy}\SpecialCharTok{$}\NormalTok{SSB}\SpecialCharTok{\textasciitilde{}}\NormalTok{yrs, }\AttributeTok{type=}\StringTok{"l"}\NormalTok{, }\AttributeTok{xlab=}\StringTok{"Year"}\NormalTok{, }\AttributeTok{ylab=}\StringTok{"Spawning Stock Biomass"}\NormalTok{,}
     \AttributeTok{las=}\DecValTok{1}\NormalTok{, }\AttributeTok{lwd=}\DecValTok{2}\NormalTok{, }\AttributeTok{main=}\StringTok{"Matts Stock"}\NormalTok{)}
\end{Highlighting}
\end{Shaded}

\pandocbounded{\includegraphics[keepaspectratio]{Population_dynamics_module_assignment_files/figure-latex/biomass-1.pdf}}

\begin{Shaded}
\begin{Highlighting}[]
\NormalTok{N\_df}\OtherTok{\textless{}{-}}\FunctionTok{data.frame}\NormalTok{(}\AttributeTok{N=}\FunctionTok{c}\NormalTok{(N),}\AttributeTok{year=}\FunctionTok{rep}\NormalTok{(yrs, nages), }\AttributeTok{age=}\NormalTok{(}\FunctionTok{rep}\NormalTok{(}\FunctionTok{seq}\NormalTok{(}\AttributeTok{from=}\DecValTok{1}\NormalTok{, }\AttributeTok{to=}\NormalTok{nages),}
                                                       \AttributeTok{each=}\DecValTok{51}\NormalTok{)))}

\FunctionTok{ggplot}\NormalTok{(}\AttributeTok{data=}\NormalTok{N\_df, }\FunctionTok{aes}\NormalTok{(}\AttributeTok{x=}\NormalTok{year, }\AttributeTok{y=}\NormalTok{N))}\SpecialCharTok{+}\FunctionTok{xlab}\NormalTok{(}\StringTok{"Year"}\NormalTok{)}\SpecialCharTok{+}\FunctionTok{ylab}\NormalTok{(}\StringTok{"Abundance at Age"}\NormalTok{)}\SpecialCharTok{+}
  \FunctionTok{geom\_line}\NormalTok{()}\SpecialCharTok{+}\FunctionTok{ggtitle}\NormalTok{(stock)}\SpecialCharTok{+}
  \FunctionTok{facet\_wrap}\NormalTok{(}\SpecialCharTok{\textasciitilde{}}\NormalTok{age, }\AttributeTok{scales =} \StringTok{"free\_y"}\NormalTok{)}\SpecialCharTok{+}
  \FunctionTok{theme}\NormalTok{(}\AttributeTok{axis.text.x =} \FunctionTok{element\_text}\NormalTok{(}\AttributeTok{angle =} \DecValTok{45}\NormalTok{, }\AttributeTok{hjust=}\DecValTok{1}\NormalTok{))}
\end{Highlighting}
\end{Shaded}

\pandocbounded{\includegraphics[keepaspectratio]{Population_dynamics_module_assignment_files/figure-latex/N_mat-1.pdf}}

\begin{Shaded}
\begin{Highlighting}[]
\NormalTok{N\_df}\OtherTok{\textless{}{-}}\FunctionTok{data.frame}\NormalTok{(}\AttributeTok{N=}\FunctionTok{c}\NormalTok{(sim\_pop\_dy}\SpecialCharTok{$}\NormalTok{N),}\AttributeTok{year=}\FunctionTok{rep}\NormalTok{(yrs, sim\_pop\_dy}\SpecialCharTok{$}\NormalTok{nages),}
                 \AttributeTok{age=}\NormalTok{(}\FunctionTok{rep}\NormalTok{(}\FunctionTok{seq}\NormalTok{(}\AttributeTok{from=}\DecValTok{1}\NormalTok{, }\AttributeTok{to=}\NormalTok{sim\_pop\_dy}\SpecialCharTok{$}\NormalTok{nages), }\AttributeTok{each=}\DecValTok{51}\NormalTok{)))}

\FunctionTok{ggplot}\NormalTok{(}\AttributeTok{data=}\NormalTok{N\_df, }\FunctionTok{aes}\NormalTok{(}\AttributeTok{x=}\NormalTok{year, }\AttributeTok{y=}\NormalTok{N))}\SpecialCharTok{+}\FunctionTok{xlab}\NormalTok{(}\StringTok{"Year"}\NormalTok{)}\SpecialCharTok{+}\FunctionTok{ylab}\NormalTok{(}\StringTok{"Abundance at Age"}\NormalTok{)}\SpecialCharTok{+}
  \FunctionTok{geom\_line}\NormalTok{()}\SpecialCharTok{+}\FunctionTok{ggtitle}\NormalTok{(}\StringTok{"Matts Stock"}\NormalTok{)}\SpecialCharTok{+}
  \FunctionTok{facet\_wrap}\NormalTok{(}\SpecialCharTok{\textasciitilde{}}\NormalTok{age, }\AttributeTok{scales =} \StringTok{"free\_y"}\NormalTok{)}\SpecialCharTok{+}
  \FunctionTok{theme}\NormalTok{(}\AttributeTok{axis.text.x =} \FunctionTok{element\_text}\NormalTok{(}\AttributeTok{angle =} \DecValTok{45}\NormalTok{, }\AttributeTok{hjust=}\DecValTok{1}\NormalTok{))}
\end{Highlighting}
\end{Shaded}

\pandocbounded{\includegraphics[keepaspectratio]{Population_dynamics_module_assignment_files/figure-latex/N_mystock-1.pdf}}

\subsection{Recruitment (2 points)}\label{recruitment-2-points}

Please describe any differences that you see between your
stock-recruitment curve and mine (1 point). What implications do these
differences have for the population dynamics of your stock relative to
mine (1 point)?

\begin{Shaded}
\begin{Highlighting}[]
\NormalTok{sim\_ssb}\OtherTok{\textless{}{-}}\FunctionTok{seq}\NormalTok{(}\AttributeTok{from=}\DecValTok{0}\NormalTok{, }\AttributeTok{to=}\FunctionTok{max}\NormalTok{(SSB), }\AttributeTok{by=}\DecValTok{1}\NormalTok{)}
\NormalTok{sim\_rec}\OtherTok{\textless{}{-}}\FunctionTok{bev\_holt}\NormalTok{(sim\_ssb, R0, SSB0, h) }

\FunctionTok{par}\NormalTok{(}\AttributeTok{mfrow=}\FunctionTok{c}\NormalTok{(}\DecValTok{2}\NormalTok{,}\DecValTok{1}\NormalTok{))}
\FunctionTok{plot}\NormalTok{(R}\SpecialCharTok{\textasciitilde{}}\NormalTok{SSB, }\AttributeTok{xlab=}\StringTok{"Spawning Stock Biomass"}\NormalTok{, }\AttributeTok{ylab=}\StringTok{"Recruitment (numbers)"}\NormalTok{,}
     \AttributeTok{las=}\DecValTok{1}\NormalTok{, }\AttributeTok{lwd=}\DecValTok{2}\NormalTok{, }\AttributeTok{pch=}\DecValTok{19}\NormalTok{, }\AttributeTok{xlim=}\FunctionTok{c}\NormalTok{(}\DecValTok{0}\NormalTok{,}\FunctionTok{max}\NormalTok{(SSB)}\SpecialCharTok{+}\DecValTok{100}\NormalTok{),}
     \AttributeTok{ylim=}\FunctionTok{c}\NormalTok{(}\DecValTok{0}\NormalTok{, }\FunctionTok{max}\NormalTok{(R[}\DecValTok{2}\SpecialCharTok{:}\NormalTok{nyears])}\SpecialCharTok{+}\DecValTok{10}\NormalTok{), }\AttributeTok{main=}\NormalTok{stock)}
\FunctionTok{lines}\NormalTok{(sim\_ssb, sim\_rec)}

\NormalTok{sim\_ssb}\OtherTok{\textless{}{-}}\FunctionTok{seq}\NormalTok{(}\AttributeTok{from=}\DecValTok{0}\NormalTok{, }\AttributeTok{to=}\FunctionTok{max}\NormalTok{(sim\_pop\_dy}\SpecialCharTok{$}\NormalTok{SSB), }\AttributeTok{by=}\DecValTok{1}\NormalTok{)}
\FunctionTok{plot}\NormalTok{(sim\_pop\_dy}\SpecialCharTok{$}\NormalTok{R}\SpecialCharTok{\textasciitilde{}}\NormalTok{sim\_pop\_dy}\SpecialCharTok{$}\NormalTok{SSB, }\AttributeTok{xlab=}\StringTok{"Spawning Stock Biomass"}\NormalTok{,}
     \AttributeTok{ylab=}\StringTok{"Recruitment (numbers)"}\NormalTok{, }\AttributeTok{las=}\DecValTok{1}\NormalTok{, }\AttributeTok{lwd=}\DecValTok{2}\NormalTok{, }\AttributeTok{pch=}\DecValTok{19}\NormalTok{,}
     \AttributeTok{xlim=}\FunctionTok{c}\NormalTok{(}\DecValTok{0}\NormalTok{,}\FunctionTok{max}\NormalTok{(sim\_pop\_dy}\SpecialCharTok{$}\NormalTok{SSB)}\SpecialCharTok{+}\DecValTok{100}\NormalTok{),}
     \AttributeTok{ylim=}\FunctionTok{c}\NormalTok{(}\DecValTok{0}\NormalTok{, }\FunctionTok{max}\NormalTok{(sim\_pop\_dy}\SpecialCharTok{$}\NormalTok{R[}\DecValTok{2}\SpecialCharTok{:}\NormalTok{nyears])}\SpecialCharTok{+}\DecValTok{100}\NormalTok{), }\AttributeTok{main=}\StringTok{"Matts Stock"}\NormalTok{)}
\FunctionTok{lines}\NormalTok{(sim\_ssb,}
\NormalTok{      (}\DecValTok{10}\SpecialCharTok{/}\FunctionTok{mean}\NormalTok{(sim\_pop\_dy}\SpecialCharTok{$}\NormalTok{maa}\SpecialCharTok{*}\NormalTok{sim\_pop\_dy}\SpecialCharTok{$}\NormalTok{waa)}\SpecialCharTok{*}\NormalTok{sim\_ssb)}\SpecialCharTok{/}\NormalTok{(}\DecValTok{1}\FloatTok{+0.001}\SpecialCharTok{*}\NormalTok{sim\_ssb))}
\end{Highlighting}
\end{Shaded}

\pandocbounded{\includegraphics[keepaspectratio]{Population_dynamics_module_assignment_files/figure-latex/sr-1.pdf}}

\subsection{Sustainability (3 points)}\label{sustainability-3-points}

Below I create a Kobe plot (using B/Bmsy vs F/Fmsy) and a USR/LRP plot
(using SSB/SSB0). What do these plots indicate about the sustainability
of the fishery on your stock?

\begin{Shaded}
\begin{Highlighting}[]
\NormalTok{B\_ratio }\OtherTok{\textless{}{-}}\NormalTok{ SSB }\SpecialCharTok{/}\NormalTok{ Bmsy}
\NormalTok{F\_ratio }\OtherTok{\textless{}{-}} \FunctionTok{rowMeans}\NormalTok{(F) }\SpecialCharTok{/}\NormalTok{ Fmsy}

\NormalTok{xmax }\OtherTok{\textless{}{-}} \FunctionTok{max}\NormalTok{(}\FloatTok{1.5}\NormalTok{, }\FunctionTok{max}\NormalTok{(B\_ratio))}
\NormalTok{ymax }\OtherTok{\textless{}{-}} \FunctionTok{max}\NormalTok{(}\FloatTok{1.5}\NormalTok{, }\FunctionTok{max}\NormalTok{(F\_ratio))}

\FunctionTok{plot}\NormalTok{(B\_ratio, F\_ratio, }\AttributeTok{type=}\StringTok{"n"}\NormalTok{, }
     \AttributeTok{xlab=}\StringTok{"B / Bmsy"}\NormalTok{, }\AttributeTok{ylab=}\StringTok{"F / Fmsy"}\NormalTok{,}
     \AttributeTok{main=}\FunctionTok{paste}\NormalTok{(}\StringTok{"Kobe Plot:"}\NormalTok{, stock),}
     \AttributeTok{xlim=}\FunctionTok{c}\NormalTok{(}\DecValTok{0}\NormalTok{, xmax), }\AttributeTok{ylim=}\FunctionTok{c}\NormalTok{(}\DecValTok{0}\NormalTok{, ymax))}

\FunctionTok{rect}\NormalTok{(}\AttributeTok{xleft=}\DecValTok{1}\NormalTok{, }\AttributeTok{ybottom=}\DecValTok{0}\NormalTok{, }\AttributeTok{xright=}\NormalTok{xmax, }\AttributeTok{ytop=}\DecValTok{1}\NormalTok{, }\AttributeTok{col=}\FunctionTok{adjustcolor}\NormalTok{(}\StringTok{"green"}\NormalTok{, }\AttributeTok{alpha.f=}\FloatTok{0.2}\NormalTok{), }\AttributeTok{border=}\ConstantTok{NA}\NormalTok{)}
\FunctionTok{rect}\NormalTok{(}\AttributeTok{xleft=}\DecValTok{0}\NormalTok{, }\AttributeTok{ybottom=}\DecValTok{0}\NormalTok{, }\AttributeTok{xright=}\DecValTok{1}\NormalTok{, }\AttributeTok{ytop=}\DecValTok{1}\NormalTok{, }\AttributeTok{col=}\FunctionTok{adjustcolor}\NormalTok{(}\StringTok{"yellow"}\NormalTok{, }\AttributeTok{alpha.f=}\FloatTok{0.2}\NormalTok{), }\AttributeTok{border=}\ConstantTok{NA}\NormalTok{)}
\FunctionTok{rect}\NormalTok{(}\AttributeTok{xleft=}\DecValTok{1}\NormalTok{, }\AttributeTok{ybottom=}\DecValTok{1}\NormalTok{, }\AttributeTok{xright=}\NormalTok{xmax, }\AttributeTok{ytop=}\NormalTok{ymax, }\AttributeTok{col=}\FunctionTok{adjustcolor}\NormalTok{(}\StringTok{"orange"}\NormalTok{, }\AttributeTok{alpha.f=}\FloatTok{0.2}\NormalTok{), }\AttributeTok{border=}\ConstantTok{NA}\NormalTok{)}
\FunctionTok{rect}\NormalTok{(}\AttributeTok{xleft=}\DecValTok{0}\NormalTok{, }\AttributeTok{ybottom=}\DecValTok{1}\NormalTok{, }\AttributeTok{xright=}\DecValTok{1}\NormalTok{, }\AttributeTok{ytop=}\NormalTok{ymax, }\AttributeTok{col=}\FunctionTok{adjustcolor}\NormalTok{(}\StringTok{"red"}\NormalTok{, }\AttributeTok{alpha.f=}\FloatTok{0.2}\NormalTok{), }\AttributeTok{border=}\ConstantTok{NA}\NormalTok{)}

\FunctionTok{abline}\NormalTok{(}\AttributeTok{v=}\DecValTok{1}\NormalTok{, }\AttributeTok{h=}\DecValTok{1}\NormalTok{, }\AttributeTok{lty=}\DecValTok{2}\NormalTok{)}

\FunctionTok{lines}\NormalTok{(B\_ratio, F\_ratio, }\AttributeTok{type=}\StringTok{"b"}\NormalTok{, }\AttributeTok{pch=}\DecValTok{19}\NormalTok{, }\AttributeTok{col=}\StringTok{"black"}\NormalTok{)}
\end{Highlighting}
\end{Shaded}

\pandocbounded{\includegraphics[keepaspectratio]{Population_dynamics_module_assignment_files/figure-latex/kobe-1.pdf}}

\begin{Shaded}
\begin{Highlighting}[]
\NormalTok{ssb\_ratio }\OtherTok{\textless{}{-}}\NormalTok{ SSB }\SpecialCharTok{/}\NormalTok{ SSB0}

\FunctionTok{plot}\NormalTok{(yrs, ssb\_ratio, }\AttributeTok{type=}\StringTok{"b"}\NormalTok{, }\AttributeTok{pch=}\DecValTok{19}\NormalTok{,}
     \AttributeTok{xlab=}\StringTok{"Year"}\NormalTok{, }\AttributeTok{ylab=}\StringTok{"SSB / SSB0"}\NormalTok{,}
     \AttributeTok{main=}\FunctionTok{paste}\NormalTok{(}\StringTok{"Stock Status:"}\NormalTok{, stock),}
     \AttributeTok{ylim=}\FunctionTok{c}\NormalTok{(}\DecValTok{0}\NormalTok{, }\FunctionTok{max}\NormalTok{(}\FloatTok{1.2}\NormalTok{, ssb\_ratio)))}

\FunctionTok{abline}\NormalTok{(}\AttributeTok{h=}\NormalTok{USR}\SpecialCharTok{/}\NormalTok{SSB0, }\AttributeTok{col=}\StringTok{"blue"}\NormalTok{, }\AttributeTok{lty=}\DecValTok{2}\NormalTok{)  }\CommentTok{\# USR}
\FunctionTok{abline}\NormalTok{(}\AttributeTok{h=}\NormalTok{LRP}\SpecialCharTok{/}\NormalTok{SSB0, }\AttributeTok{col=}\StringTok{"red"}\NormalTok{, }\AttributeTok{lty=}\DecValTok{2}\NormalTok{)   }\CommentTok{\# LRP}

\FunctionTok{legend}\NormalTok{(}\StringTok{"topright"}\NormalTok{,}
       \AttributeTok{legend=}\FunctionTok{c}\NormalTok{(}\StringTok{"USR"}\NormalTok{, }\StringTok{"LRP"}\NormalTok{),}
       \AttributeTok{col=}\FunctionTok{c}\NormalTok{(}\StringTok{"black"}\NormalTok{,}\StringTok{"red"}\NormalTok{),}
       \AttributeTok{lty=}\FunctionTok{c}\NormalTok{(}\DecValTok{2}\NormalTok{,}\DecValTok{2}\NormalTok{),}
       \AttributeTok{pch=}\FunctionTok{c}\NormalTok{(}\ConstantTok{NA}\NormalTok{,}\ConstantTok{NA}\NormalTok{))}
\end{Highlighting}
\end{Shaded}

\pandocbounded{\includegraphics[keepaspectratio]{Population_dynamics_module_assignment_files/figure-latex/usr_lrp-1.pdf}}

\end{document}
